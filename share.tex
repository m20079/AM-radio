\documentclass[report.tex]{subfiles}

\externaldocument{report}

\begin{document}

\section{役割分担}

\begin{table}[H]
	\centering
	\caption{役割分担}
	\label{tab:yakuwari}
	\begin{tabular}{lccc} \hline
		役割        & 竹中 & 野村 & 片田 \\ \hline
		レポート目標    & ○  & ○  & ×  \\
		レポート原理・設計 & ○  & ×  & ×  \\
		レポート性能評価  & ○  & ×  & ×  \\ \hline
	\end{tabular}
\end{table}

\subsection{竹中}

\begin{itemize}
	\item レポート(1 目標、2 原理・設計、3 性能評価、4 失敗点、5 役割分担の図と文章全部)
	\item 産技祭展示用ポスターの作成
	\item アンテナ1号機、2号機の作成
	\item アンプ1号機、2号機、3号機、4号機(手伝い)、5号機の設計
	\item アンプ1号機、2号機、3号機、4号機、5号機の修繕
	\item 回路の大まかな設計
	\item アンテナの設計
	\item レポートを書くための回路の測定(アンプ、共振回路)
	\item 班員の人たちに役割分担をする
	\item 基板加工機を用いた基盤への穴あけ加工
	\item 共振周波数のシミュレーション
	\item 増幅回路のシミュレーション
	\item 買い出し一回目、二回目、三回目、四回目、五回目
\end{itemize}

\subsection{野村}

\begin{itemize}
	\item レポート(1 目標、5 役割分担)
	\item アンテナ1号の作成
	\item 先生に実験室や機材室の鍵を開けてもらうように頼む役割
	\item フィルタ回路の設計、制作
	\item 4号機ラジオアンプ回路設計、制作
	\item 基板加工機を用いた基盤への穴あけ加工
	\item スペーサ取り付け
	\item 基盤、エンドミルの調達
	\item 共振周波数、Q値算出
	\item 買い出し一回目、二回目、三回目
\end{itemize}

\subsection{片田}

\begin{itemize}
	\item レポート(\wfig{map}の作成、5 役割分担)
	\item 産技祭展示用ポスターの作成
	\item アンプ一台目の基板、五代目の基板のハンダ付け
	\item 機材室からアンプを作成するための材料を取ってくる
	\item アンテナ1号機、2号機の作成
	\item 電灯線アンテナの作成
	\item 増幅回路の素子の調達
	\item amラジオ送信元の調査
	\item レポートを書くための回路の測定(共振回路)
	\item 買い出し一回目、二回目
\end{itemize}

\end{document}
