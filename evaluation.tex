\documentclass[report.tex]{subfiles}

\externaldocument{report}

\begin{document}

\section{性能評価}

% 野村の担当
対して,今回製作できたラジオは,以下のような仕様となった。
\begin{enumerate}
  \item ゲルマニウムダイオードを用いた検波を行うことができる。
  \item 4段のアンプを用いてスピーカーを駆動することができる。
  \item AMニッポン放送(1242k\,Hz)のみ受信することができる。
  \item 通すアンプの段数を変えることにより音量調整をすることができる。
  \item 直径240mmのループアンテナを有し,持ち運びできない。
  \item 向きを考慮すれば,およそCAD室内のどこでも受信できる。
\end{enumerate}

上記項目が得られた以外に,定量的な測定を行った。
\subsection{アンテナ}
今回,合計で3つのアンテナを用意した(1つは既製品,二つは製作したもの)。それぞれの仕様を\wtab{ant}に示す。

\begin{table}[h]
	\centering
	\caption{アンテナの各パラメータ}
	\label{tab:ant}
	\begin{tabular}{ccccc} \hline
		
		名称&直径&巻き数&コア材質&コア長\\ \hline \hline
		一号機&240&24&空気&-\\
		二号機&120&160&空気&-\\
		三号機&10&―&フェライト&80\\ \hline

	\end{tabular}
\end{table}

これらのアンテナに対し我々が行った評価の項目は,以下の2つである。
\begin{enumerate}
  \item 受信強度
  \item 指向性
\end{enumerate}

\subsubsection{受信強度}
アンテナをラジオ回路に接続したときの,クリスタルイヤホンからの音量によって,受信強度を評価した。一号機ではかすかに聞こえる(内容の把握はまれに可能である)ほどであり,二号機,三号機ではほとんど聞こえなかった。
\subsubsection{指向性}
上記回路で,アンテナを置く位置,向きを変更したときの,クリスタルイヤホンからの音量によって,指向性を評価した。一号機に関する結果を述べる。CAD室の窓際に置き,送信局へ向けた時最大の音量となった。それ以外の条件ではほとんど聞こえなかった。二号機,三号機に関しては評価を実施していない。

\subsection{ラジオ回路}
ラジオ回路の評価を難航した。

\subsection{アンプ}
\end{document}
