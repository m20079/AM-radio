\documentclass[report.tex]{subfiles}

\externaldocument{report}

\begin{document}

\section{性能評価}

今回製作できたラジオは,以下のような仕様となった。

\begin{enumerate}
	\item ゲルマニウムダイオードを用いた検波を行うことができた。
	\item 4段のアンプを用いてスピーカーを駆動することができた。
	\item 多少の雑音はあるが、何を話しているのかを聞き取ることができた。
	\item AMニッポン放送(1242\,kHz)しか受信することが出来なかった。
	\item 可変抵抗を用いて音量調整を行うことができた。
	\item 直径24cm、高さ38cmのループアンテナを用いてるため、持ち運びが困難であった。
	\item およそCAD室内のどこでも受信できた。
\end{enumerate}

\subsection{アンテナ}

今回は合計で3つのアンテナを用意した(1号機と2号機は製作したもの、3号機は既製品)。
それぞれの仕様を\wtab{ant}に示す。
また、それぞれの写真を\wfig{1},\wfig{2},\wfig{3}に示す。

\begin{table}[h]
	\centering
	\caption{アンテナの各パラメータ}
	\label{tab:ant}
	\begin{tabular}{ccccccc} \hline
		名称  & 直径[mm] & 巻き数[turn] & 抵抗[\(\Omega\)] & 長さ[mm] & コア材質  & コア長[mm] \\ \hline
		1号機 & 240    & 24        & 4.6            & 11     & 空気    & -       \\
		2号機 & 65     & 135       & 0.5            & 50     & 空気    & -       \\
		3号機 & 10     & 70        & 28.0           & 12     & フェライト & 42      \\ \hline
	\end{tabular}
\end{table}

\begin{figure}[H]
	\centering
	\includegraphics[width=7cm]{fig/1.jpg}
	\caption{1号機アンテナの写真}
	\label{fig:1}
\end{figure}

\begin{figure}[H]
	\begin{minipage}[b]{0.5\linewidth}
		\centering
		\includegraphics[width=7cm]{fig/2.jpg}
		\caption{2号機アンテナの写真}
		\label{fig:2}
	\end{minipage}
	\begin{minipage}[b]{0.5\linewidth}
		\centering
		\includegraphics[width=7cm]{fig/3.jpg}
		\caption{3号機アンテナの写真}
		\label{fig:3}
	\end{minipage}
\end{figure}

それぞれのインダクタンスの測定は、\wfig{inda}のような回路を用いて行った。
周波数を変更させて、共振回路間の電圧を測定し、分圧則より共振周波数の場合はインピーダンスがほぼ無限になるため、5Vの電圧が測定できる。
5Vになった周波数から、\weq{resonance}と共振回路内のセラミックコンデンサ150 pFの値を用いて、インダクタンスを計算した。

\begin{figure}[H]
	\centering
	\includegraphics[width=10cm]{fig/inda.pdf}
	\caption{インダクタンス測定回路}
	\label{fig:inda}
\end{figure}

\wfig{inda2}に、\wfig{inda}の測定結果を示す。
1号機(Unit 1)と2号機(Unit2)は5.00 Vが計測できたが、3号機(Unit3)は4.31 Vまでしか計測できなかった。
また、\wfig{inda2}には載せていないが、3号機のフェライトを抜いたら共振しなくなり、0 Vが観測された。

\begin{figure}[H]
	\centering
	\includegraphics[width=10cm]{fig/1_kyo.pdf}
	\caption{インダクタンス測定回路の結果}
	\label{fig:inda2}
\end{figure}

\wfig{inda2}の結果から、\wtab{ant2}のように、インダクタンスと受信できる周波数を計算した。
受信できる周波数は、4 pF \(\sim\) 260 pFのバリアブルコンデンサを用いることを想定している。

\begin{table}[h]
	\centering
	\caption{インダクタンス測定回路}
	\label{tab:ant2}
	\begin{tabular}{ccccc} \hline
		名称  & 最大電圧[V] & 共振周波数[kHz] & インダクタンス[\(\upmu\)H] & 受信できる周波数[kHz]           \\ \hline
		1号機 & 5.00    & 約645       & 405.91              & 489.91 \(\sim\) 3949.80 \\
		2号機 & 5.00    & 約390       & 1110.24             & 296.23 \(\sim\) 2388.26 \\
		3号機 & 4.31    & 約790       & 270.58              & 600.69 \(\sim\) 4842.93 \\ \hline
	\end{tabular}
\end{table}

実際に、それぞれのアンテナで音を聞くと、

\begin{itemize}
	\item 1号機は1242 kHzのニッポン放送ははっきり聞こえるが
\end{itemize}

\subsubsection{受信強度}
アンテナをラジオ回路に接続したときの,クリスタルイヤホンからの音量によって,受信強度を評価した。一号機ではかすかに聞こえる(内容の把握はまれに可能である)ほどであり,二号機,三号機ではほとんど聞こえなかった。
\subsubsection{指向性}
上記回路で,アンテナを置く位置,向きを変更したときの,クリスタルイヤホンからの音量によって,指向性を評価した。一号機に関する結果を述べる。CAD室の窓際に置き,送信局へ向けた時最大の音量となった。それ以外の条件ではほとんど聞こえなかった。二号機,三号機に関しては評価を実施していない。

\subsection{ラジオ回路}
ラジオ回路の評価は難航した。回路にかかる電圧は微小信号であり,オシロスコープや電圧計での測定ができない。そのため,アンプで増幅してから評価するか,やはり出力された音声による評価を行う他なかった。

ラジオ回路に対し我々が行った評価は,アンテナ→ラジオ回路→スピーカーという風に出力したときの音声に含まれるノイズを

\subsection{アンプ}

\wfig{gain}に、今回用いた増幅回路のゲイン線図を示す。
\wfig{gain}のTheoretical valueが理想値(5.7倍、15.1175[dB])である。
Lowest rangeが人間が聞こえる周波数の最低値(20[Hz])である。
Highest rangeが人間が聞こえる周波数の最高値(20000[Hz])である。
測定値は、オシロスコープから取り出した波形のCSVファイルから最大値を取り出したものなので、
ノイズなどでゲインが少し理想値よりも高くなっている。
\(10^6\)[Hz]付近で、リップルが発生しており、その後に急激に減少している。
しかし、この領域は人間が聞こえる周波数の範囲外のため、問題ないと考えられる。
逆に、人間が聞こえる周波数の範囲内は、理想値に近い値を取っているため良いと考えられる。

\begin{figure}[H]
	\centering
	\includegraphics[width=10cm]{fig/gain.pdf}
	\caption{アンプのゲイン線図}
	\label{fig:gain}
\end{figure}

\end{document}
